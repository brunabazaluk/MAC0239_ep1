\documentclass[12pt]{article}
 
\usepackage[margin=1in]{geometry} 
\usepackage{amsmath,amsthm,amssymb}
 
\newcommand{\N}{\mathbb{N}}
\newcommand{\Z}{\mathbb{Z}}

\begin{document}

\title{\textbf{EP 1 - Sudoku}}%replace X with the appropriate number
\author{Bruna Bazaluk M Videira\\ %replace with your name
\\ nUSP: 9797002
\\ \textbf{MAC0239 - Introducao a Logica e Verificacao de Programas}} %if necessary, replace with your course title

\maketitle


\large{}

Minha ideia foi codificar as regras do sudoku e transforma-las em CNFs utilizando mapas de Karnaugh. As regras codificadasforam:
\begin{itemize}
\item Um unico valor por posicao
\item Um unico valor por linha
\item Um unico valor por coluna
\item Um unico valor por quadrante
\end{itemize}

Como sao 4 opcoes para valores, considerei a sentenca: $(p \wedge -q \wedge -r \wedge -s) \lor (-p \wedge q \wedge -r \wedge -s) \lor (-p \wedge -q \wedge r \wedge -s) \lor (-p \wedge -q \wedge -r \wedge s) $ e para cada regra, p, q, r e s teriam significados diferentes. Porem, a expressao acima nao e uma CNF, entao, apos aplicar o mapa de Karnaugh, cheguei a seguinte expressao: $ (p \lor q \lor r \lor s) \wedge (~p \lor ~q) \wedge  (~p \lor ~r)  (~p \lor ~s)  (~q \lor ~r)  (~q \lor ~s)  (~r \lor ~s)  $, que foi aplicada em meu codigo.

\end{document}
